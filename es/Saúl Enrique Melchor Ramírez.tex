%%%%%%%%%%%%%%%%%%%%%%%%%%%%%%%%%%%%%%%%%
% Developer CV
% LaTeX Template
% Version 1.0 (28/1/19)
%
% This template originates from:
% http://www.LaTeXTemplates.com
%
% Authors:
% Jan Vorisek (jan@vorisek.me)
% Based on a template by Jan Küster (info@jankuester.com)
% Modified for LaTeX Templates by Vel (vel@LaTeXTemplates.com)
%
% License:
% The MIT License (see included LICENSE file)
%
%%%%%%%%%%%%%%%%%%%%%%%%%%%%%%%%%%%%%%%%%

%----------------------------------------------------------------------------------------
%	PACKAGES AND OTHER DOCUMENT CONFIGURATIONS
%----------------------------------------------------------------------------------------

\documentclass[9pt]{developercv} % Default font size, values from 8-12pt are recommended

%----------------------------------------------------------------------------------------

\begin{document}

%----------------------------------------------------------------------------------------
%	TITLE AND CONTACT INFORMATION
%----------------------------------------------------------------------------------------

\begin{minipage}[t]{0.45\textwidth} % 45% of the page width for name
	\vspace{-\baselineskip} % Required for vertically aligning minipages
	
	% If your name is very short, use just one of the lines below
	% If your name is very long, reduce the font size or make the minipage wider and reduce the others proportionately
	\colorbox{black}{{\HUGE\textcolor{white}{\textbf{\MakeUppercase{Saúl E.}}}}} % First name
	
    \colorbox{black}{{\HUGE\textcolor{white}{\textbf{\MakeUppercase{Melchor}}}}} % Last name
    
    \colorbox{black}{{\HUGE\textcolor{white}{\textbf{\MakeUppercase{Ramírez}}}}} % Last name
	
	\vspace{6pt}
	
	{\huge Ingeniero de Software} % Career or current job title
\end{minipage}
\begin{minipage}[t]{0.55\textwidth} % 27.5% of the page width for the first row of icons
	\vspace{-\baselineskip} % Required for vertically aligning minipages
	
	% The first parameter is the FontAwesome icon name, the second is the box size and the third is the text
	% Other icons can be found by referring to fontawesome.pdf (supplied with the template) and using the word after \fa in the command for the icon you want
	\icon{MapMarker}{12}{Xalapa, Ver. México}\\
	\icon{Phone}{12}{+52 228 2310 336}\\
    \icon{At}{12}{\href{mailto:saulenriquemr@outlook.com}{saulenriquemr@outlook.com}}\\
    \icon{Github}{12}{\href{https://github.com/saulenriquemr}{github.com/saulenriquemr}}\\
\end{minipage}

\vspace{0.5cm}

%----------------------------------------------------------------------------------------
%	EDUCATION
%----------------------------------------------------------------------------------------

\cvsect{Educación}

\begin{entrylist}
	\entry
		{11/2019}
		{Curso}
		{Universidad Veracruzana}
        {Curso: Uso de Angular como herramienta para el desarrollo de aplicaciones web profesionales. Duración: 30 horas.\\
        \texttt{Angular 8}\slashsep\texttt{Angular Material}\slashsep\texttt{NGBootstrap}}
	\entry
		{8/2019}
		{Maestría}
		{LANIA}
        {Mestría en Redes y Sistemas Integrados, con titulación prevista para agosto 2021.\\
        \texttt{Java}\slashsep\texttt{Spring Framework}\slashsep\texttt{Vue.js}\slashsep\texttt{Oracle Database}\slashsep\texttt{Administración de redes}}
	\entry
		{8/2014 -- 8/2018}
		{Licenciatura}
		{Universidad Veracruzana}
        {Carrera: Ingeniería de Software\\
        Monografía: Métricas de Estabilidad de Software\\
        Cédula profesional: 11917758\\
        \texttt{Requerimientos}\slashsep\texttt{Diseño}\slashsep\texttt{Documentación}\slashsep\texttt{Desarrollo}\slashsep\texttt{Pruebas}\slashsep\texttt{Despliegue}}
\end{entrylist}

%----------------------------------------------------------------------------------------
%	PUBLICATIONS
%----------------------------------------------------------------------------------------

\cvsect{Publicaciones}

\begin{entrylist}
	\entry
		{2018}
		{Software Stability: A Systematic Literature Review}
		{CONISOFT}
        {Revisión sistemática de la literatura acerca de la Estabilidad de Software, presentada en la 6\textsuperscript{a} edición de la Conferencia Internacional de Investigación e Innovación en Ingeniería de Software.\\{\href{https://ieeexplore.ieee.org/document/8645866}{DOI: 10.1109/CONISOFT.2018.8645866}}\\
        \texttt{Mendeley}\slashsep\texttt{Zotero}}
\end{entrylist}


%----------------------------------------------------------------------------------------
%	EXPERIENCE
%----------------------------------------------------------------------------------------

\cvsect{Experiencia}

\begin{entrylist}
	\entry
		{8/2021 -- 1/2022}
		{Profesor}
		{Universidad Veracruzana}
        {Impartiendo la Experiencia Educativa de Desarrollo de Sistemas en Red perteneciente al Programa Educativo de Ingeniería de Software en la Universidad Veracruzana.\\
        \texttt{Java}\slashsep\texttt{ASP.NET Core}\slashsep\texttt{NodeJS}}
	\entry
		{1/2021}
		{Desarrollador full-stack}
		{We-ai}
		{Desarrollo de módulo PRET, desarrollo de módulo para facturación y cancelación de CFDIs versión 3.3 y 4.0. Administración de CDN.\\
		\texttt{Docker}\slashsep\texttt{VueJS 3}\slashsep\texttt{PostgreSQL}\slashsep\texttt{Django 3.2}\slashsep\texttt{AWS}}
    \entry
		{2/2020 -- 7/2020}
		{Profesor}
		{Universidad Veracruzana}
        {Impartiendo la Experiencia Educativa de Desarrollo de Aplicaciones perteneciente al Programa Educativo de Ingeniería de Software en la Universidad Veracruzana.\\
        \texttt{Docker}\slashsep\texttt{Android}\slashsep\texttt{ASP.NET Core}}
	\entry
		{1/2020 -- 12/2020}
		{Desarrollador front-end}
		{Universidad Veracruzana}
        {Desarrollo y Despliegue de recursos dígitales de aprendizaje con base en un prototipos. En la Dirección de Extensión de Servicios Tecnológicos perteneciente a la Universidad Veracruzana.
        \texttt{HTML}\slashsep\texttt{SCSS}\slashsep\texttt{Vue.js}\slashsep\texttt{MDBootstrap}}
	\entry
		{1/2019 -- 12/2019}
		{Desarrollador full-stack}
		{Universidad Veracruzana}
        {En el programa \textit{ALAS - Alfabetizar a Sordos}, aquí se desarrolló: módulo de seguimiento de usuarios, módulo de almacenamiento de imágenes, correcciones al portal principal y una sección para el aprendizaje de enunciados. \\
        \texttt{ASP.NET}\slashsep\texttt{SQL Server}\slashsep\texttt{Vue.js}\slashsep\texttt{HTML}\slashsep\texttt{SCSS}\slashsep\texttt{Bootstrap}\slashsep\texttt{Wordpress}}
	\entry
		{8/2018 -- 12/2018}
		{Desarrollador front-end}
		{Universidad Veracruzana}
        {Desarrollo y despliegue de recursos dígitales de aprendizaje con base en un prototipos para la plataforma. \textit{CODAES}. \\
        \texttt{HTML}\slashsep\texttt{CSS}\slashsep\texttt{Bootstrap}\slashsep\texttt{JQuery}\slashsep\texttt{SCORM}}
\end{entrylist}

%----------------------------------------------------------------------------------------
%	ADDITIONAL INFORMATION
%----------------------------------------------------------------------------------------

% \begin{minipage}[t]{0.3\textwidth}
% 	\vspace{-\baselineskip} % Required for vertically aligning minipages

% 	\cvsect{Languages}
	
% 	\textbf{English} - native\\
% 	\textbf{German} - proficient\\
% 	\textbf{Polish} - rudimentary
% \end{minipage}
% \hfill
% \begin{minipage}[t]{0.3\textwidth}
% 	\vspace{-\baselineskip} % Required for vertically aligning minipages
	
% 	\cvsect{Hobbies}
	
% 	I love... \lorem
% \end{minipage}
% \hfill
% \begin{minipage}[t]{0.3\textwidth}
% 	\vspace{-\baselineskip} % Required for vertically aligning minipages
	
% 	\cvsect{Non profit}
	
% 	I help... \lorem
% \end{minipage}

%----------------------------------------------------------------------------------------

\end{document}
